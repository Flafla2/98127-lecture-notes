%
% 98-127
% Lecture 03: Wrangling Unity
% 
% Template adapted from
% https://www.cs.cmu.edu/~15150/previous-semesters/2012-spring/resources/latex/template.tex
%

\documentclass[11pt]{article}

\usepackage{amsmath}
\usepackage{amssymb}
\usepackage{amsthm}
\usepackage{hyperref}
\usepackage{fancyhdr}
\usepackage{listings}
\usepackage{color}
\usepackage{graphicx}
\usepackage{setspace}
\usepackage{times}
\usepackage{wrapfig}
\usepackage{fontawesome}

\usepackage{menukeys}

\oddsidemargin0cm
\topmargin-2cm
\textwidth16.5cm
\textheight23.5cm

\newtheorem{lemma}{Lemma}

%% Begin Constants %% <-- (edit these!)

% Lecture Info
\newcommand{\lecturenum}{3}
\newcommand{\lecturename}{Wrangling Unity}

% Author of THIS document
\newcommand{\authorname}{Adrian Biagioli}

% Course Info
\newcommand{\coursenum}{98-127}
\newcommand{\coursename}{Game Creation for People Who Want to Make Games}
\newcommand{\coursesem}{S19}

% Instructors of course
\newcommand{\instructors}{Adrian Biagioli (\,\href{mailto:abiagiol@andrew.cmu.edu}{abiagiol@andrew.cmu.edu}\,) \\ Carter Williams (\,\href{mailto:ncwillia@andrew.cmu.edu}{ncwillia@andrew.cmu.edu}\,)}

% Math shortcuts
\newcommand{\work}{\mathcal{W}}
\newcommand{\expv}{\mathbf{E}}
\newcommand{\pr}{\mathbf{Pr}}
\newcommand{\ts}{\textsuperscript}
\newcommand{\bigo}{\mathit{O}}

%% End Constants %%

%% Begin syntax highlighting settings %%

\definecolor{dkgreen}{rgb}{0,0.6,0}
\definecolor{gray}{rgb}{0.5,0.5,0.5}
\definecolor{mauve}{rgb}{0.58,0,0.82}

\lstset{frame=tb,
    language={[Sharp]C},
    aboveskip=3mm,
    belowskip=3mm,
    showstringspaces=false,
    columns=flexible,
    basicstyle={\small\ttfamily},
    numbers=left,
    stepnumber=1,
    numberstyle=\tiny\color{gray},
    keywordstyle=\color{blue},
    commentstyle=\color{dkgreen},
    stringstyle=\color{mauve},
    breaklines=true,
    breakatwhitespace=true,
    tabsize=3
  }

\lstnewenvironment{csharp}[1][]
{
  \lstset{frame=tb,
    language={[Sharp]C},
    aboveskip=3mm,
    belowskip=3mm,
    showstringspaces=false,
    columns=flexible,
    basicstyle={\small\ttfamily},
    numbers=left,
    stepnumber=1,
    numberstyle=\tiny\color{gray},
    keywordstyle=\color{blue},
    commentstyle=\color{dkgreen},
    stringstyle=\color{mauve},
    breaklines=true,
    breakatwhitespace=true,
    tabsize=3
  }
}{}

\lstnewenvironment{pseudocode}[1][]
{
    \lstset{
        mathescape=true,
        frame=tb,
        numbers=left, 
        basicstyle=\small, 
        numberstyle=\tiny\color{gray},
        keywordstyle=\color{black}\bfseries\em,
        commentstyle=\color{dkgreen},
        stringstyle=\color{mauve},
        keywords={,input, output, return, datatype, function, in, if, else, foreach, while, begin, end, }
        numbers=left,
        breaklines=true,
        breakatwhitespace=true,
        tabsize=3
    }
}
{}

% From https://tex.stackexchange.com/questions/95036/continue-line-numbers-in-listings-package
\def\ContinueLineNumber{\lstset{firstnumber=last}}
\def\StartLineAt#1{\lstset{firstnumber=#1}}
\let\numberLineAt\StartLineAt

%% End syntax highlighting settings %%

\setlength{\parindent}{2em}
\setlength{\parskip}{5pt plus 1pt}
\renewcommand{\baselinestretch}{1.15}

\pagestyle{fancyplain}
{
    \lhead{\fancyplain{}{Lecture \lecturenum}}
    \rhead{\fancyplain{}{\coursenum}}
    \chead{\fancyplain{}{\lecturename}}
}
\setlength{\headheight}{14pt}

\graphicspath{ {./images/} }

\begin{document}

\thispagestyle{plain}
{
    \vspace{1.5em}
    \begin{center}
    {
        \huge
        Lecture \lecturenum \\
        \vspace{0.5em}
        \lecturename
        \vspace{0.4em}
    } \\
    {
        \it
        \coursenum: \coursename\ \ (\coursesem)
    } \\
    \vspace{1.0em}
    Written by \authorname \\
    \vspace{0.7em}
    Instructors:\\ \instructors
    \end{center}
}

\section{Objectives}

By the end of this lesson you will be able to:
\begin{itemize}
    \item Create C\# scripts in the Unity editor, edit them, and apply them to GameObjects as Components
    \item Understand the basics of Unity's \lstinline|MonoBehaviour|: \lstinline|Start()|, \lstinline|Update()|, and more
    \item Add easy-to-use interfaces to your component in the Unity Inspector, enabling artists and level designers to
          tweak parameters in your Unity code.
    \item Allow your scripts to inter-operate with other Components (of your own creation or Unity's)
    \item Use Prefabs to dynamically create GameObjects in your scripts
\end{itemize}

\noindent These lecture notes were written for {\bf Unity 2018.3.3f1}.

\section{Downloading the Base Code}

You can download all of the base code for this lecture via the following Unitypackage.  In these lecture notes, I've
included the path to each script before every code sample (for example, the first code sample below is labeled
\directory{BasicFunctions/StartTester.cs}, which means that you can find \lstinline|StartTester.cs| in the
\lstinline|BasicFunctions| folder).

\begin{center}
\href{http://stage.gamecreation.org/StuCo/packages/lec03resources.unitypackage}{http://stage.gamecreation.org/StuCo/packages/lec03resources.unitypackage}
\end{center}

\noindent See the lecture 2 notes for more info on how to import Unitypackages.

\pagebreak

\section{Cooking Components}

In the last lecture, we went over how to use components created by your team members or Unity themselves to compose
complex relationships between GameObjects.  But what if we want to define our own custom behavior?  Unity allows you
to create your own components via a \textbf{C\# Script}.  C\# is a programming language maintained by Microsoft.  It is
very similar to Java and should feel familiar to anyone who knows a C-based language.  We will not cover how to code
in this class; if you are interested in learning more about using C\#, you can find plenty of resources online.

To create a new component type, you need to create a C\# Script in the project view.  Open up the project view in Unity,
right click on it, and select \menu{Create > C\# Script}.  Alternatively you can navigate to \menu{Assets > Create >
C\# Script}.  Read more about C\# \href{https://docs.microsoft.com/en-us/dotnet/csharp/}{here}.  Name your first script
\lstinline|StartTester.cs|.  Let's make this script print the familiar ``Hello, World'' to the debugging console.  Type
the following in the script:

\directory{BasicFunctions/StartTester.cs}
\StartLineAt{1}
\begin{csharp}
using UnityEngine;

public class StartTester : MonoBehaviour
{
    // Start() is called exactly once when you launch the game
    private void Start()
    {
        // Use Debug.Log(...) to log to the Console view
        Debug.Log("Hello, World!");
    }
}
\end{csharp}


\directory{BasicFunctions/UpdateTester.cs}
\StartLineAt{1}
\begin{csharp}
using UnityEngine;

public class UpdateTester : MonoBehaviour
{
    // Update() is called once per frame
    void Update()
    {
        Debug.Log("Hello, World! (a lot)");
    }
}
\end{csharp}

\directory{BasicFunctions/FieldTester.cs}
\StartLineAt{1}
\begin{csharp}
using System.Collections;
using System.Collections.Generic;
using UnityEngine;

public class FieldTester : MonoBehaviour
{
    // Mark fields with [SerializeField] to allow them to be edited in the inspector
    [SerializeField]
    private string StringToPrint;

    private void Start()
    {
        Debug.Log(StringToPrint);
    }
}
\end{csharp}

\pagebreak

\directory{BasicFunctions/LotsOfFieldsTester.cs}
\StartLineAt{1}
\begin{csharp}
using UnityEngine;
using UnityEngine.Events;

public class LotsOfFieldsTester : MonoBehaviour
{
    [Header("Basic Fields")]
    [SerializeField]
    private int _IntField;
    [SerializeField]
    private float _FloatField = 5.0f;
    [SerializeField]
    private string _StringField = "Test Field";
    [SerializeField]
    private Vector3 _VectorField = new Vector3(42, 69, 1337);

    [Header("Object Fields")]
    [SerializeField]
    private GameObject _GameObjectField;
    [SerializeField]
    private Rigidbody _ComponentField1;
    [SerializeField]
    private MeshRenderer _ComponentField2;
    [SerializeField]
    private Material _AssetField1;
    [SerializeField]
    private Mesh _AssetField2;

    [Header("Basic Fields with Controls")]
    [SerializeField]
    [Range(-1.0f, 1.0f)]
    private float _RangeFloatField;
    [SerializeField]
    [TextArea]
    private string _LargeStringField;

    [Header("Weird/Advanced Fields")]
    [SerializeField]
    private UnityEvent _EventField;
    [SerializeField]
    private LayerMask _LayerMaskField;
}
\end{csharp}

\directory{BasicFunctions/ClosedFormAnimation.cs}
\StartLineAt{1}
\begin{csharp}
using UnityEngine;

public class ClosedFormAnimation : MonoBehaviour
{
    [SerializeField]
    private float Radius = 1.0f;
    [SerializeField]
    private Vector3 Center = Vector3.zero; // Vector3.zero == new Vector3(0,0,0)
    [SerializeField]
    private float Speed = 1.0f;

    private void Update()
    {
        float s = Mathf.Sin(Time.time * Speed);
        float c = Mathf.Cos(Time.time * Speed);

        transform.position = Center + new Vector3(c * Radius, s * Radius, 0);
    }
}
\end{csharp}

\directory{BasicFunctions/DeltaTimeAnimation.cs}
\StartLineAt{1}
\begin{csharp}
using UnityEngine;

public class DeltaTimeAnimation : MonoBehaviour
{
    // You don't need [SerializeField] for public variables
    public float Speed = 1.0f; // speed in m/s
    public Vector3 Direction = new Vector3(1,0,0);
    public bool localPosition = false;

    // Update is called once per frame
    private void Update()
    {
        Vector3 delta = Speed * Direction.normalized;

        // This doesn't work!!  What is wrong?
        if (localPosition)
            transform.localPosition += delta;
        else
            transform.position += delta;
    }
}
\end{csharp}

\directory{BasicFunctions/SimpleMovement.cs}
\StartLineAt{1}
\begin{csharp}
using UnityEngine;

public class SimpleMovement : MonoBehaviour
{
    [SerializeField]
    private string _HorizontalMovementAxis = "Horizontal";
    [SerializeField]
    private string _VerticalMovementAxis = "Vertical";
    [SerializeField]
    private float _MovementSpeed = 1.0f;

    private void Update()
    {
        float hoz = Input.GetAxis(_HorizontalMovementAxis);
        float vrt = Input.GetAxis(_VerticalMovementAxis);

        Vector3 mov = new Vector3(hoz, vrt, 0);

        if (mov.sqrMagnitude > 1.0f)
            mov.Normalize(); // make vector have length 1

        transform.position += mov * Time.deltaTime;
    }
}
\end{csharp}

\directory{BasicFunctions/ChangeMaterialTester.cs}
\StartLineAt{1}
\begin{csharp}
using UnityEngine;

[RequireComponent(typeof(MeshRenderer))]
public class ChangeMaterialTester : MonoBehaviour
{
    [SerializeField]
    private Gradient _Gradient;
    [SerializeField]
    [Tooltip("Length in seconds to cycle through the gradient")]
    private float _CycleLength;

    // Note: This field isn't serialized, so we can't edit it in the inspector!
    private MeshRenderer _Renderer;

    private void Start()
    {
        // GetComponent<T> gets the component with type T attached to the current GameObject
        // If none exist, returns null
        _Renderer = GetComponent<MeshRenderer>();
    }

    private void Update()
    {
        float a = (Mathf.Cos(Time.time / _CycleLength * (2.0f * Mathf.PI)) + 1.0f) / 2.0f;
        Color c = _Gradient.Evaluate(a);

        _Renderer.material.color = c;
    }
}
\end{csharp}

\section{Physics Examples}

\textbf{FIRST: } Make sure to talk about Physics materials and Colliders!

\directory{PhysicsExamples/CollisionDetectionTester.cs}
\StartLineAt{1}
\begin{csharp}
using UnityEngine;

public class CollisionDetectionTester : MonoBehaviour
{
    private void OnTriggerEnter(Collider other)
    {
        Debug.Log("Just got triggered by GameObject called " + other.gameObject.name);
    }

    private void OnTriggerExit(Collider other)
    {
        Debug.Log("Just ended trigger by GameObject called " + other.gameObject.name);
    }

    private void OnCollisionEnter(Collision collision)
    {
        Debug.Log("Just collided with GameObject called " + collision.gameObject.name);
    }

    private void OnCollisionExit(Collision collision)
    {
        Debug.Log("Just ended collision with GameObject called " + collision.gameObject.name);
    }
}
\end{csharp}

\directory{PhysicsExamples/CollisionDetectionTester2D.cs}
\StartLineAt{1}
\begin{csharp}
using UnityEngine;

public class CollisionDetectionTester2D : MonoBehaviour
{
    private void OnTriggerEnter2D(Collider2D other)
    {
        Debug.Log("Just got triggered by GameObject called " + other.gameObject.name);
    }

    private void OnTriggerExit2D(Collider2D other)
    {
        Debug.Log("Just ended trigger by GameObject called " + other.gameObject.name);
    }

    private void OnCollisionEnter2D(Collision2D collision)
    {
        Debug.Log("Just collided with GameObject called " + collision.gameObject.name);
    }

    private void OnCollisionExit2D(Collision2D collision)
    {
        Debug.Log("Just ended collision with GameObject called " + collision.gameObject.name);
    }
}
\end{csharp}

\directory{PhysicsExamples/EventOnTrigger.cs}
\StartLineAt{1}
\begin{csharp}
using UnityEngine;
using UnityEngine.Events;

public class EventOnTrigger : MonoBehaviour
{
    [SerializeField]
    private UnityEvent _OnEnter;
    [SerializeField]
    private UnityEvent _OnExit;

    private void OnTriggerEnter(Collider other)
    {
        if(_OnEnter != null)
            _OnEnter.Invoke();
    }

    private void OnTriggerExit(Collider other)
    {
        if(_OnExit != null)
            _OnExit.Invoke();
    }
}
\end{csharp}

\directory{PhysicsExamples/EventOnLookAndPress.cs}
\StartLineAt{1}
\begin{csharp}
// Don't write this script in front of the class.  But you can show it being used!
using UnityEngine;
using UnityEngine.Events;

public class EventOnLookAndPress : MonoBehaviour
{
    [SerializeField]
    private string _ButtonAxis = "Fire1"; // default to left click
    [SerializeField]
    private float _ReachDistance = 5;
    [SerializeField]
    private UnityEvent _Event;

    private void Update()
    {
        if (_Event == null)
            return;

        Camera cam = Camera.main;
        if (cam == null) // no main camera
            return;

        // GetButtonDown is ONLY true on the frame you pressed the button
        if(Input.GetButtonDown(_ButtonAxis))
        {
            RaycastHit hit;
            if(Physics.Raycast(cam.transform.position, cam.transform.forward, out hit, _ReachDistance))
            {
                if (hit.collider.gameObject == gameObject)
                    _Event.Invoke();
            }
        }
    }
}
\end{csharp}

\section{Prefabs}

\textbf{FIRST: } Introduce Prefabs using the Prefab example scene

\directory{PrefabExamples/SpawnPrefabOnButton.cs}
\StartLineAt{1}
\begin{csharp}
using UnityEngine;

public class SpawnPrefabOnButton : MonoBehaviour
{
    [SerializeField]
    private string _ButtonToPress = "Fire1";
    [SerializeField]
    private GameObject _PrefabToSpawn;
    [SerializeField]
    private Transform _TransformToSpawnAt;

    private void Update()
    {
        if(Input.GetButtonDown(_ButtonToPress))
        {
            var go = Instantiate<GameObject>(_PrefabToSpawn);
            go.transform.position = _TransformToSpawnAt == null ? transform.position : _TransformToSpawnAt.position;
            go.transform.rotation = _TransformToSpawnAt == null ? transform.rotation : _TransformToSpawnAt.rotation;
        }
    }
}
\end{csharp}

\directory{PrefabExamples/SpawnPrefabInterval.cs}
\StartLineAt{1}
\begin{csharp}
using System.Collections;
using UnityEngine;

public class SpawnPrefabInterval : MonoBehaviour
{
    [SerializeField]
    private float _SpawnIntervalSeconds = 1.0f;
    [SerializeField]
    private GameObject _PrefabToSpawn;
    [SerializeField]
    private Transform _TransformToSpawnAt;

    private void Start()
    {
        StartCoroutine(SpawnPrefabCoroutine());
    }

    private IEnumerator SpawnPrefabCoroutine()
    {
        while(true)
        {
            var go = Instantiate<GameObject>(_PrefabToSpawn);
            go.transform.position = _TransformToSpawnAt == null ? transform.position : _TransformToSpawnAt.position;
            go.transform.rotation = _TransformToSpawnAt == null ? transform.rotation : _TransformToSpawnAt.rotation;

            yield return new WaitForSeconds(_SpawnIntervalSeconds);
        }
    }
}
\end{csharp}




\end{document}