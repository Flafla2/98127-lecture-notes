%
% 98-127
% Lecture 09: Advanced Topics: Graphics and Lighting
% 
% Template adapted from
% https://www.cs.cmu.edu/~15150/previous-semesters/2012-spring/resources/latex/template.tex
%

\documentclass[11pt]{article}

\usepackage{amsmath}
\usepackage{amssymb}
\usepackage{amsthm}
\usepackage{hyperref}
\usepackage{fancyhdr}
\usepackage{listings}
\usepackage{color}
\usepackage{graphicx}
\usepackage{setspace}
\usepackage{times}
\usepackage{wrapfig}
\usepackage{fontawesome}

\usepackage{menukeys}

\oddsidemargin0cm
\topmargin-2cm
\textwidth16.5cm
\textheight23.5cm

\newtheorem{lemma}{Lemma}

%% Begin Constants %% <-- (edit these!)

% Lecture Info
\newcommand{\lecturenum}{9}
\newcommand{\lecturename}{Advanced Topics: Graphics and Lighting}

% Author of THIS document
\newcommand{\authorname}{Adrian Biagioli}

% Course Info
\newcommand{\coursenum}{98-127}
\newcommand{\coursename}{Game Creation for People Who Want to Make Games}
\newcommand{\coursesem}{S19}

% Instructors of course
\newcommand{\instructors}{Adrian Biagioli (\,\href{mailto:abiagiol@andrew.cmu.edu}{abiagiol@andrew.cmu.edu}\,) \\ Carter Williams (\,\href{mailto:ncwillia@andrew.cmu.edu}{ncwillia@andrew.cmu.edu}\,)}

% Math shortcuts
\newcommand{\work}{\mathcal{W}}
\newcommand{\expv}{\mathbf{E}}
\newcommand{\pr}{\mathbf{Pr}}
\newcommand{\ts}{\textsuperscript}
\newcommand{\bigo}{\mathit{O}}

%% End Constants %%

%% Begin syntax highlighting settings %%

\definecolor{dkgreen}{rgb}{0,0.6,0}
\definecolor{gray}{rgb}{0.5,0.5,0.5}
\definecolor{mauve}{rgb}{0.58,0,0.82}

\lstset{frame=tb,
    language={[Sharp]C},
    aboveskip=3mm,
    belowskip=3mm,
    showstringspaces=false,
    columns=flexible,
    basicstyle={\small\ttfamily},
    numbers=left,
    stepnumber=1,
    numberstyle=\tiny\color{gray},
    keywordstyle=\color{blue},
    commentstyle=\color{dkgreen},
    stringstyle=\color{mauve},
    breaklines=true,
    breakatwhitespace=true,
    tabsize=3
  }

\lstnewenvironment{csharp}[1][]
{
  \lstset{frame=tb,
    language={[Sharp]C},
    aboveskip=3mm,
    belowskip=3mm,
    showstringspaces=false,
    columns=flexible,
    basicstyle={\small\ttfamily},
    numbers=left,
    stepnumber=1,
    numberstyle=\tiny\color{gray},
    keywordstyle=\color{blue},
    commentstyle=\color{dkgreen},
    stringstyle=\color{mauve},
    breaklines=true,
    breakatwhitespace=true,
    tabsize=3
  }
}{}

\lstnewenvironment{pseudocode}[1][]
{
    \lstset{
        mathescape=true,
        frame=tb,
        numbers=left, 
        basicstyle=\small, 
        numberstyle=\tiny\color{gray},
        keywordstyle=\color{black}\bfseries\em,
        commentstyle=\color{dkgreen},
        stringstyle=\color{mauve},
        keywords={,input, output, return, datatype, function, in, if, else, foreach, while, begin, end, }
        numbers=left,
        breaklines=true,
        breakatwhitespace=true,
        tabsize=3
    }
}
{}

% From https://tex.stackexchange.com/questions/95036/continue-line-numbers-in-listings-package
\def\ContinueLineNumber{\lstset{firstnumber=last}}
\def\StartLineAt#1{\lstset{firstnumber=#1}}
\let\numberLineAt\StartLineAt

%% End syntax highlighting settings %%

\setlength{\parindent}{2em}
\setlength{\parskip}{5pt plus 1pt}
\renewcommand{\baselinestretch}{1.15}

\pagestyle{fancyplain}
{
    \lhead{\fancyplain{}{Lecture \lecturenum}}
    \rhead{\fancyplain{}{\coursenum}}
    \chead{\fancyplain{}{\lecturename}}
}
\setlength{\headheight}{14pt}

\graphicspath{ {./images/} }

\begin{document}

\thispagestyle{plain}
{
    \vspace{1.5em}
    \begin{center}
    {
        \huge
        Lecture \lecturenum \\
        \vspace{0.5em}
        \lecturename
        \vspace{0.4em}
    } \\
    {
        \it
        \coursenum: \coursename\ \ (\coursesem)
    } \\
    \vspace{1.0em}
    Written by \authorname \\
    \vspace{0.7em}
    Instructors:\\ \instructors
    \end{center}
}

\section{Objectives}

By the end of this lesson you will be able to:
\begin{itemize}
    \item Understand Unity's Material system, and the difference between Shaders and Materials
    \item Understand the concepts of Physically Based Rendering (PBR) as well as the inputs to Unity's Standard PBR shader
    \item Bake environment lighting into a texture to support large statically-lit scenes.
    \item Understand Unity's lighting settings to balance visual fidelity and performance.
\end{itemize}

\noindent These lecture notes were written for {\bf Unity 2019.2.6}.

\section{What are Materials?}

Up to this point in the course, we have discussed how to load 3D models into your Unity scene as a GameObject, and how to attach behaviors to those GameObjects.  However we have not yet discussed {\it how} those objects are drawn on the screen.  For example, what color do we shade an object at each point?  In a previous lecture, we discussed how to create a UV map of a 3D model to wrap a texture onto it.  Indeed, we can use a texture and a UV map to color an object.  

\begin{enumerate}
	\item Creating a material in the project view
	\item Material vs Shader
	\item Manipulating Material Properties (textures, colors)
\end{enumerate}

\section{The Standard Shader}

\begin{enumerate}
	\item Discuss the standard shader from a high level -- maybe talk about the history of shaders (lambert / blinn based specularity) and what motivates the Standard shader.  \textbf{Define PBR (Physically Based Rendering)}
	\item Big idea: The standard shader is a "meta-shader" that is able to express photoreal surfaces.  Use custom shaders for non-photoreal stuff
	\item Go over the material charts.  Big idea: Albedo $\neq$ Color!  Especially show effect of albedo on metals
	\item To demonstrate PBR reflections, demonstrate changing the skybox to an HDR one
\end{enumerate}

\section{Finding PBR Textures}

\begin{enumerate}
	\item Find a texture from CC0Textures.com and a HDR from HDRIhaven.com, and show how to import all the maps (via GIMP)
	\item Smoothness vs Roughness
	\item Define HDR images (not a simple color, but an \textit{irradiance})
	\item Demonstrate reflection probes in the room scene
\end{enumerate}

\section{Unity's Lighting panel and Global Illumination}

\begin{enumerate}
	\item Explain why we need to bake lighting (performance + complex prebaked lighting)
	\item Bake lighting in the room scene
	\item Put a big bright green cube in the middle of the scene and bake again
	\item Discuss static / lightmap static
	\item Point lights only support baked shadows
\end{enumerate}


\end{document}