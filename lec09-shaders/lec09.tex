%
% 98-127
% Lecture 09: Advanced Topics: Graphics and Lighting
% 
% Template adapted from
% https://www.cs.cmu.edu/~15150/previous-semesters/2012-spring/resources/latex/template.tex
%

\documentclass[11pt]{article}

\usepackage{amsmath}
\usepackage{amssymb}
\usepackage{amsthm}
\usepackage{hyperref}
\usepackage{fancyhdr}
\usepackage{listings}
\usepackage{color}
\usepackage{graphicx}
\usepackage{setspace}
\usepackage{times}
\usepackage{wrapfig}
\usepackage{fontawesome}

\usepackage{menukeys}

\oddsidemargin0cm
\topmargin-2cm
\textwidth16.5cm
\textheight23.5cm

\newtheorem{lemma}{Lemma}

%% Begin Constants %% <-- (edit these!)

% Lecture Info
\newcommand{\lecturenum}{9}
\newcommand{\lecturename}{Advanced Topics: Graphics and Lighting}

% Author of THIS document
\newcommand{\authorname}{Adrian Biagioli}

% Course Info
\newcommand{\coursenum}{98-127}
\newcommand{\coursename}{Game Creation for People Who Want to Make Games}
\newcommand{\coursesem}{S19}

% Instructors of course
\newcommand{\instructors}{Adrian Biagioli (\,\href{mailto:abiagiol@andrew.cmu.edu}{abiagiol@andrew.cmu.edu}\,) \\ Carter Williams (\,\href{mailto:ncwillia@andrew.cmu.edu}{ncwillia@andrew.cmu.edu}\,)}

% Math shortcuts
\newcommand{\work}{\mathcal{W}}
\newcommand{\expv}{\mathbf{E}}
\newcommand{\pr}{\mathbf{Pr}}
\newcommand{\ts}{\textsuperscript}
\newcommand{\bigo}{\mathit{O}}

%% End Constants %%

%% Begin syntax highlighting settings %%

\definecolor{dkgreen}{rgb}{0,0.6,0}
\definecolor{gray}{rgb}{0.5,0.5,0.5}
\definecolor{mauve}{rgb}{0.58,0,0.82}

\lstset{frame=tb,
    language={[Sharp]C},
    aboveskip=3mm,
    belowskip=3mm,
    showstringspaces=false,
    columns=flexible,
    basicstyle={\small\ttfamily},
    numbers=left,
    stepnumber=1,
    numberstyle=\tiny\color{gray},
    keywordstyle=\color{blue},
    commentstyle=\color{dkgreen},
    stringstyle=\color{mauve},
    breaklines=true,
    breakatwhitespace=true,
    tabsize=3
  }

\lstnewenvironment{csharp}[1][]
{
  \lstset{frame=tb,
    language={[Sharp]C},
    aboveskip=3mm,
    belowskip=3mm,
    showstringspaces=false,
    columns=flexible,
    basicstyle={\small\ttfamily},
    numbers=left,
    stepnumber=1,
    numberstyle=\tiny\color{gray},
    keywordstyle=\color{blue},
    commentstyle=\color{dkgreen},
    stringstyle=\color{mauve},
    breaklines=true,
    breakatwhitespace=true,
    tabsize=3
  }
}{}

\lstnewenvironment{pseudocode}[1][]
{
    \lstset{
        mathescape=true,
        frame=tb,
        numbers=left, 
        basicstyle=\small, 
        numberstyle=\tiny\color{gray},
        keywordstyle=\color{black}\bfseries\em,
        commentstyle=\color{dkgreen},
        stringstyle=\color{mauve},
        keywords={,input, output, return, datatype, function, in, if, else, foreach, while, begin, end, }
        numbers=left,
        breaklines=true,
        breakatwhitespace=true,
        tabsize=3
    }
}
{}

% From https://tex.stackexchange.com/questions/95036/continue-line-numbers-in-listings-package
\def\ContinueLineNumber{\lstset{firstnumber=last}}
\def\StartLineAt#1{\lstset{firstnumber=#1}}
\let\numberLineAt\StartLineAt

%% End syntax highlighting settings %%

\setlength{\parindent}{2em}
\setlength{\parskip}{5pt plus 1pt}
\renewcommand{\baselinestretch}{1.15}

\pagestyle{fancyplain}
{
    \lhead{\fancyplain{}{Lecture \lecturenum}}
    \rhead{\fancyplain{}{\coursenum}}
    \chead{\fancyplain{}{\lecturename}}
}
\setlength{\headheight}{14pt}

\graphicspath{ {./images/} }

\begin{document}

\thispagestyle{plain}
{
    \vspace{1.5em}
    \begin{center}
    {
        \huge
        Lecture \lecturenum \\
        \vspace{0.5em}
        \lecturename
        \vspace{0.4em}
    } \\
    {
        \it
        \coursenum: \coursename\ \ (\coursesem)
    } \\
    \vspace{1.0em}
    Written by \authorname \\
    \vspace{0.7em}
    Instructors:\\ \instructors
    \end{center}
}

\section{Objectives}

By the end of this lesson you will be able to:
\begin{itemize}
    \item Understand Unity's Material system, and the difference between Shaders and Materials
    \item Understand the concepts of Physically Based Rendering (PBR) as well as the inputs to Unity's Standard PBR shader
    \item Use Unity's Effects pipeline to compose complex post processing effects
    \item Bake lightmaps onto Unity scenes to improve performance
    \item Understand the basics of creating your own shaders via Unity's experimental Shader Graph system and Surface shaders
\end{itemize}

\noindent These lecture notes were written for {\bf Unity 2018.3.8}.

\section{What are Materials?}

\begin{enumerate}
	\item Creating a material in the project view
	\item Material vs Shader
	\item Manipulating Material Properties (textures, colors)
\end{enumerate}

\section{The Standard Shader}

\begin{enumerate}
	\item Discuss the standard shader from a high level -- maybe talk about the history of shaders (lambert / blinn based specularity) and what motivates the Standard shader.  \textbf{Define PBR (Physically Based Rendering)}
	\item Big idea: The standard shader is a "meta-shader" that is able to express photoreal surfaces.  Use custom shaders for non-photoreal stuff
	\item Go over the material charts.  Big idea: Albedo $\neq$ Color!  Especially show effect of albedo on metals
	\item To demonstrate PBR reflections, demonstrate changing the skybox to an HDR one
\end{enumerate}

\section{Finding PBR Textures}

\begin{enumerate}
	\item Find a texture from CC0Textures.com and a HDR from HDRIhaven.com, and show how to import all the maps (via GIMP)
	\item Smoothness vs Roughness
	\item Define HDR images (not a simple color, but an \textit{irradiance})
	\item Demonstrate reflection probes in the room scene
\end{enumerate}

\section{Unity's Lighting panel and Global Illumination}

\begin{enumerate}
	\item Explain why we need to bake lighting (performance + complex prebaked lighting)
	\item Bake lighting in the room scene
	\item Put a big bright green cube in the middle of the scene and bake again
	\item Discuss static / lightmap static
	\item Point lights only support baked shadows
\end{enumerate}


\section{Adding Post-processing image effects}

\begin{enumerate}
	\item How to download post process stack (package manager, or asset store for v1)
	\item Adding post process layers to the camera
	\item Adding specific effects via a post process volume (have to create post process profile first)
	\item Motion blur makes things look better for free :)
	\item Remember to set the layer to the corresponding layer on the camera (show them how)
\end{enumerate}


\section{Writing custom Shaders: The surface shader}

\begin{enumerate}
	\item Surface Shader examples: https://docs.unity3d.com/Manual/SL-SurfaceShaderExamples.html
	\item Surface Shader output struct: https://docs.unity3d.com/Manual/SL-SurfaceShaders.html
	\item Write a basic rim lighting shader that writes to emission (also define emission if you havent) and then show the surface shader example page
\end{enumerate}

Rim lighting example from unity docs:
\begin{csharp}
Shader "Example/Rim" {
	Properties {
	  _MainTex ("Texture", 2D) = "white" {}
	  _BumpMap ("Bumpmap", 2D) = "bump" {}
	  _RimColor ("Rim Color", Color) = (0.26,0.19,0.16,0.0)
	  _RimPower ("Rim Power", Range(0.5,8.0)) = 3.0
	}
	SubShader {
	  Tags { "RenderType" = "Opaque" }
	  CGPROGRAM
	  #pragma surface surf Lambert
	  struct Input {
	      float2 uv_MainTex;
	      float2 uv_BumpMap;
	      float3 viewDir;
	  };
	  sampler2D _MainTex;
	  sampler2D _BumpMap;
	  float4 _RimColor;
	  float _RimPower;
	  void surf (Input IN, inout SurfaceOutput o) {
	      o.Albedo = tex2D (_MainTex, IN.uv_MainTex).rgb;
	      o.Normal = UnpackNormal (tex2D (_BumpMap, IN.uv_BumpMap));
	      half rim = 1.0 - saturate(dot (normalize(IN.viewDir), o.Normal));
	      o.Emission = _RimColor.rgb * pow (rim, _RimPower);
	  }
	  ENDCG
	} 
	Fallback "Diffuse"
}
\end{csharp}

\section{Advanced: Using the Unity Shader Graph}

\begin{enumerate}
	\item Talk briefly about the scriptable render pipelines and how they are the future of unity rendering
	\item Briefly show how to install the new pipelines (remember to enable experimental packages in the package manager)
	\item Do the rim lighting example again, this time using shader graphs
\end{enumerate}

\section{Case Study: Legend of Zelda lighting}

\begin{enumerate}
	\item Discuss the toon shading, briefly discuss NDotL lighting and how smoothstep is important here
	\item Discuss how the eyes are rendered: Start by demonstrating the shadergraph version of the pupil shader on a plane, then show the raw shader version (this will be the first time they see a vert/fragment shader).  Talk about the depth buffer and how you can ignore it for artistic effects.  Mention render order input in materials.
	\item Show the LOZ BOTW shadergraph example: https://connect.unity.com/p/zelda-inspired-toon-shading-in-shadergraph
\end{enumerate}


\end{document}